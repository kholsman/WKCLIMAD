% Options for packages loaded elsewhere
\PassOptionsToPackage{unicode}{hyperref}
\PassOptionsToPackage{hyphens}{url}
%
\documentclass[
]{article}
\title{WKCLIMAD Network Models}
\author{}
\date{\vspace{-2.5em}}

\usepackage{amsmath,amssymb}
\usepackage{lmodern}
\usepackage{iftex}
\ifPDFTeX
  \usepackage[T1]{fontenc}
  \usepackage[utf8]{inputenc}
  \usepackage{textcomp} % provide euro and other symbols
\else % if luatex or xetex
  \usepackage{unicode-math}
  \defaultfontfeatures{Scale=MatchLowercase}
  \defaultfontfeatures[\rmfamily]{Ligatures=TeX,Scale=1}
\fi
% Use upquote if available, for straight quotes in verbatim environments
\IfFileExists{upquote.sty}{\usepackage{upquote}}{}
\IfFileExists{microtype.sty}{% use microtype if available
  \usepackage[]{microtype}
  \UseMicrotypeSet[protrusion]{basicmath} % disable protrusion for tt fonts
}{}
\makeatletter
\@ifundefined{KOMAClassName}{% if non-KOMA class
  \IfFileExists{parskip.sty}{%
    \usepackage{parskip}
  }{% else
    \setlength{\parindent}{0pt}
    \setlength{\parskip}{6pt plus 2pt minus 1pt}}
}{% if KOMA class
  \KOMAoptions{parskip=half}}
\makeatother
\usepackage{xcolor}
\IfFileExists{xurl.sty}{\usepackage{xurl}}{} % add URL line breaks if available
\IfFileExists{bookmark.sty}{\usepackage{bookmark}}{\usepackage{hyperref}}
\hypersetup{
  pdftitle={WKCLIMAD Network Models},
  hidelinks,
  pdfcreator={LaTeX via pandoc}}
\urlstyle{same} % disable monospaced font for URLs
\usepackage[margin=1in]{geometry}
\usepackage{graphicx}
\makeatletter
\def\maxwidth{\ifdim\Gin@nat@width>\linewidth\linewidth\else\Gin@nat@width\fi}
\def\maxheight{\ifdim\Gin@nat@height>\textheight\textheight\else\Gin@nat@height\fi}
\makeatother
% Scale images if necessary, so that they will not overflow the page
% margins by default, and it is still possible to overwrite the defaults
% using explicit options in \includegraphics[width, height, ...]{}
\setkeys{Gin}{width=\maxwidth,height=\maxheight,keepaspectratio}
% Set default figure placement to htbp
\makeatletter
\def\fps@figure{htbp}
\makeatother
\setlength{\emergencystretch}{3em} % prevent overfull lines
\providecommand{\tightlist}{%
  \setlength{\itemsep}{0pt}\setlength{\parskip}{0pt}}
\setcounter{secnumdepth}{-\maxdimen} % remove section numbering
\ifLuaTeX
  \usepackage{selnolig}  % disable illegal ligatures
\fi

\begin{document}
\maketitle

{
\setcounter{tocdepth}{2}
\tableofcontents
}
\begin{verbatim}
## Warning: package 'visNetwork' was built under R version 4.1.3
\end{verbatim}

\begin{verbatim}
## Warning: package 'networkD3' was built under R version 4.1.3
\end{verbatim}

\begin{verbatim}
## Warning: package 'network' was built under R version 4.1.3
\end{verbatim}

\begin{verbatim}
## Warning: package 'tidygraph' was built under R version 4.1.3
\end{verbatim}

\begin{verbatim}
## Warning: package 'ggraph' was built under R version 4.1.3
\end{verbatim}

\begin{verbatim}
## Warning: package 'viridis' was built under R version 4.1.3
\end{verbatim}

\begin{verbatim}
## Warning: package 'gdtools' was built under R version 4.1.3
\end{verbatim}

\begin{verbatim}
## Warning: package 'hrbrthemes' was built under R version 4.1.3
\end{verbatim}

\begin{verbatim}
## Warning: package 'shinyjs' was built under R version 4.1.3
\end{verbatim}

\begin{verbatim}
## Warning: package 'magick' was built under R version 4.1.3
\end{verbatim}

\hypertarget{wkclimad-overview}{%
\section{1. WKCLIMAD Overview}\label{wkclimad-overview}}

This repository contains R code and Rdata files for working with
WKCLIMAD data and responses. WKCLIMAD is an ICES workshop aimed at
exploring how can the short-, medium-, and long-term influences of
climate change on aquaculture, fisheries, and ecosystems be accounted
for in ICES Advice.

More information about WKCLIMAD can be found
here:\url{https://www.ices.dk/community/groups/Pages/WKCLIMAD.aspx}

\hypertarget{visualizing-the-results}{%
\section{2. Visualizing the results}\label{visualizing-the-results}}

The interactive shiny() is downloadable by entering the following lines
of code into R().

\hypertarget{chord-diagram}{%
\subsection{3. Chord Diagram}\label{chord-diagram}}

\hypertarget{force-diagram}{%
\subsection{4. Force Diagram}\label{force-diagram}}

\hypertarget{sankey-diagram}{%
\subsection{5. Sankey Diagram}\label{sankey-diagram}}

\end{document}
